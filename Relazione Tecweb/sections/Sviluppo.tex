\section{Sviluppo}

\vspace{8px}

\subsection{Fase di progettazione}
Durante il primo incontro del gruppo si sono delineate le linee guida da seguire nello sviluppo del progetto, tra cui la tipologia di sito da realizzare ed una prima suddivisione macroscopica dei compiti individuali.\\
Nello specifico si è deciso di impostare un layout a tre pannelli standard suddiviso in:
\begin{itemize}
 \item header \& breadcrumbs
 \item sidebar per il menù di navigazione
 \item contenuto informativo della pagina
\end{itemize}

\vspace{8px}

\subsection{Principi di design}
Il gruppo ha posto particolare attenzione al tema della completa separazione tra struttura, presentazione e comportamento. In prima battuta quindi ci si è focalizzati unicamente sullo sviluppo della struttura in XHTML Strict, codificando uno scheletro di pagine statiche sul quale, solo in seguito, si sono applicati gli opportuni layout in css puri.\\
Un altro aspetto fortemente ricercato dal gruppo è quello dell'accessibilità del sito per ogni categoria d'utenza (maggiori dettagli sull'accessibilità nel paragrafo \ref{sec:access})

\pagebreak

\subsection{Html e Css}
Oltre alla struttura di base dell'Html, un accento è stato posto sui \emph{tag meta} come \emph{keywords} e \emph{description}, utili per porre la base ad un eventuale futuro lavoro di SEO\footnote{Search Engine Optimization}. Il gruppo ha posto attenzione anche a rispettare il principio \emph{"dal particolare al generale"} nei titoli di ogni pagina, per evitare più possibile il disorientamento dell'utente all'interno sito.\\
I css sono stati sviluppati per ottenere un design fluido e funzionale per la grande maggioranza dei browser web e dispositivi. Tutte le grandezze sono state espresse in unità relative (\emph{em} o \%), fatta eccezione per i \emph{border}, espressi in \emph{px}.

% \vspace{8px}

\subsection{MySql e Php}
Il database MySql è stato sviluppato per tenere traccia degli utenti registrati e delle prenotazioni/noleggi effettuati dagli stessi. Il DBMS usato è MariaDb, l'engine di storage scelto è InnoDb, previa valutazione del buon compromesso tra dimensioni massime (64Tb sono ritenuti più che sufficienti per questa realtà) e la sicurezza nel mantenere la consistenza dei dati.\\
Sono stati sviluppati triggers e procedure per:
\begin{itemize}
\item Effettuare un noleggio di uno strumento
\item Prenotare una sala
\item Eliminare un utente
\end{itemize}
L'uso che è stato fatto di php ha lo scopo di far interagire l'utenza del sito con il database, perpetrando però i dovuti controlli sulla consistenza dei dati inseriti e/o richiesti. Php inoltre è stato sfruttato per la generazione dinamica delle pagine, ottenendo un comportamento delle stesse coerente al fatto che sia stato effettuato o meno un login\footnote{è stato tenuto conto dei permessi differenti tra utenti ed amministratori} o un'interrogazione al database.

\pagebreak

\subsection{Javascript}
