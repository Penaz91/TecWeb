\section{Sviluppo}

\vspace{8px}

\subsection{Fase di progettazione}
Durante il primo incontro del gruppo si sono delineate le linee guida da seguire nello sviluppo del progetto, tra cui la tipologia di sito da realizzare ed una prima suddivisione macroscopica dei compiti individuali.\\
Nello specifico si è deciso di impostare un layout a tre pannelli leggermente modificato rispetto allo standard, ovvero:
\begin{itemize}
 \item Header che non copre tutta la larghezza della pagina
 \item Sidebar per il menù di navigazione che comprende anche il logo del sito e che non scorre insieme al resto della pagina
 \item Breadcrumbs che scorrono insieme al contenuto della pagina ma quando ne raggiungono l'estremità superiore diventano fisse. Questa funzionalità è stata implementata per rendere sempre chiaro all'utente in che punto del sito web si trovi. Questo comportamento è disponibile solo nei browser più moderni (chrome 1.0+, firefox 1.0+, safari 6.1+ con estensione webkit sticky, opera 4.0+ e IE/Edge 16+), nei browser più datati le breadcrumbs scorrono normalmente
\end{itemize}

\vspace{8px}

\subsection{Principi di design}
Il gruppo ha posto particolare attenzione al tema della completa separazione tra struttura, presentazione e comportamento. In prima battuta quindi ci si è focalizzati unicamente sullo sviluppo della struttura in XHTML Strict, codificando uno scheletro di pagine statiche sul quale, solo in seguito, si sono applicati gli opportuni layout in css puri. Tali layout in css sono stati sviluppati per ottenere un design fluido e funzionale per la grande maggioranza dei web browsers e dispositivi. Tramite opportune \emph{media queries} si sono resi disponibili un layout standard per alte risoluzioni, un layout mobile (per smartphone e tablet) ed un layout di stampa. \\
Un principio fondamentale che il gruppo si è prodigato a rispettare è la cosiddetta regola \emph{"dal particolare al generale"} nei titoli di ogni pagina, per evitare più possibile il disorientamento dell'utente all'interno sito.\\
Un altro aspetto fortemente ricercato dal gruppo è quello dell'accessibilità del sito per ogni categoria d'utenza (maggiori dettagli sull'accessibilità nel paragrafo \ref{sec:access}).

\pagebreak

\subsection{Html e Css}
Oltre alla struttura di base dell'Html, un accento è stato posto sui \emph{tag meta} come \emph{keywords} e \emph{description}, utili per porre la base ad un eventuale futuro lavoro di SEO\footnote{Search Engine Optimization}. \\
Il menù per il mobile è stato realizzato in css simulando un menù \emph{"ad hamburger"} standard, con l'intento di scongiurare eventuali incompatibilità con dispositivi sprovvisti di supporto a Javascript. Tutte le grandezze sono state espresse in unità relative (\emph{em} o \%), fatta eccezione per i \emph{border}, espressi in \emph{px}, ed il layout di stampa.

% \vspace{8px}

\subsection{MySql e Php}
Il database MySql è stato sviluppato per tenere traccia delle sale, degli strumenti a disposizione, degli utenti registrati e delle prenotazioni/noleggi effettuati dagli stessi. Il DBMS usato è MariaDb, l'engine di storage scelto è InnoDb, previa valutazione del buon compromesso tra dimensioni massime (64Tb sono ritenuti più che sufficienti per questa realtà) e la sicurezza nel mantenere la consistenza dei dati.\\
Sono stati sviluppati triggers e procedure per:
\begin{itemize}
\item Noleggio di uno strumento (con verifica dell'effettiva disponibilità)
\item Prenotazione di una sala (con verifica dell'effettiva disponibilità)
\item Eliminazione di un utente
\end{itemize}
L'uso che è stato fatto di php ha lo scopo di far interagire l'utenza del sito con il database, perpetrando però i dovuti controlli sulla consistenza dei dati inseriti e/o richiesti. Php inoltre è stato sfruttato per la generazione dinamica delle pagine, ottenendo un comportamento delle stesse coerente al fatto che sia stato effettuato o meno un login\footnote{è stato tenuto conto dei permessi differenti tra utenti ed amministratori} o un'interrogazione al database.

\pagebreak

\subsection{Javascript}
Ci si è avvalsi di Javascript per implementare le seguenti funzionalità:
\begin{itemize}
\item Controllo sul livello di sicurezza delle password alla registrazione di un utente
\item Verifica se l'inserimento della password scelta corrisponde al reinserimento successivo durante la registrazione di un utente
\item Controllo sul corretto formato d'inserimento di date, orari, indirizzi email e numeri di telefono
\item Lightboxes caricate dinamicamente in background (asincrone), in modo da mostrare le immagini in overlay se cliccate, invece di caricarle separatamente. In caso di mancato supporto a javascript è stata implementata anche un'alternativa per mantenere il sito consistente
\item Settaggio dinamico dei placeholders
\end{itemize}