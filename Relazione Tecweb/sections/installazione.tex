\section{Installazione}
In questa sezione si specificano le modalità di installazione del sito.

\subsection{Requisiti}
\subsubsection{Applicazioni Installate}
I requisiti minimi per un corretto funzionamento di questo progetto sono:
\begin{itemize}
\item \textbf{Un server HTTP} - Si consiglia l'uso di Apache
\item \textbf{PHP versione 7.0 o superiore}
\item \textbf{Un motore database MySQL-compatible} - Noi abbiamo usato MariaDB
\end{itemize}
\subsubsection{Configurazione}
Per poter funzionare correttamente è necessario assicurarsi che alcune configurazioni siano corrette:
\begin{itemize}
\item Il server deve essere configurato in modo che PHP supporti l'upload di files (Solitamente attivo di default)
\item L'estensione "mysqli" di PHP deve essere attiva (alcune distribuzioni di Linux la disattivano di default)
\item PHP deve avere supporto XML (testato con LibXML2), per il supporto a DOMDocument (Solitamente attivo di default)
\end{itemize}

\pagebreak

\subsection{Installazione Base}
\subsubsection{La struttura delle cartelle}
Il progetto è inserito in un archivio compresso contenente la seguente struttura, a cui faremo riferimento per il resto della sezione:
\dirtree{%
.1 /.
.2 relazione.pdf.
.2 sito.
.3 CSS.
.3 Images.
.3 \ldots.
.2 config.
.3 htaccess.
.2 database.
.3 Database.sql.
.3 EliminaUtente.sql.
.3 \ldots.
}
\subsubsection{Costruzione del database}
All'interno della cartella "database" vengono forniti i file necessari alla costruzione del database attivo da cui il sito web andrà a scrivere e leggere informazioni, oltre ad un dump che sarà utile al primo popolamento.\\
La costruzione può avvenire tramite linea di comando o interfaccia PhpMyAdmin.
\paragraph{Tramite Linea di comando (Scelta consigliata)}
Dopo aver acceduto al database dalla cartella "database" è possibile creare le tabelle tramite i comandi \lstinline[language=sql]{SOURCE} che seguono:
\begin{lstlisting}[language=sql, frame=single]
        SOURCE Database.sql;
        SOURCE TriggersTW.sql;
        SOURCE EliminaUtente.sql;
        SOURCE NuovaPrenotazione.sql;
        SOURCE NuovoNoleggio.sql;
        SOURCE VerificaDisponibilita.sql;
        SOURCE Dump.sql;
\end{lstlisting}
In caso di problemi potrebbe essere necessario controllare che MySQL abbia accesso ai file di cui sopra.
\paragraph{Tramite PhpMyAdmin}
In questo caso si consiglia di prendere i file del paragrafo precedente, e copiarne il contenuto all'interno dell'interprete SQL di PhpMyAdmin, in quanto un'importazione massiva non garantirebbe di avere un database ben formato e funzionante.
\subsubsection{Installazione del sito}
Per installare il sito è sufficiente copiare l'intero contenuto della cartella "sito" all'interno della sezione del server dedicata alle pagine HTML (solitamente è \texttt{public\_html} in Apache).
\subsection{Installazione Avanzata}
\subsubsection{Le pagine di errore (Opzionale - Solo Apache)}
Allo scopo di migliorare l'accessibilità del sito si mettono a disposizione due pagine di errore personalizzate, ma per attivarle è necessario eseguire un semplice passaggio.\\
Nell'archivio del progetto, sottocartella "config", si mette a disposizione un file "htaccess" che dovrà essere messo nella root del server e rinominato ".htaccess" (attenzione al punto iniziale) allo scopo di configurare Apache per usare le pagine d'errore che abbiamo creato.\\
\textbf{Attenzione:} Su sistemi Linux i file inizianti con il carattere punto sono considerati nascosti e non potrebbero essere visibili in una struttura cartelle normale, è comunque possibile vederli tramite il comando terminale \texttt{ls -a} oppure abilitando la visualizzazione dei file nascosti nel proprio file manager.
