\section{Installazione}
In questa sezione si specificano le modalità di installazione del sito.

\subsection{Requisiti}
\subsubsection{Applicazioni Installate}
I requisiti minimi per un corretto funzionamento di questo progetto sono:
\begin{itemize}
\item \textbf{Un server HTTP} - Si consiglia l'uso di Apache
\item \textbf{PHP versione 7.0 o superiore}
\item \textbf{Un motore database MySQL-compatible} - Noi abbiamo usato MariaDB
\end{itemize}
\subsubsection{Configurazione}
Per poter funzionare correttamente è necessario assicurarsi che alcune configurazioni siano corrette:
\begin{itemize}
\item Il server deve essere configurato in modo che PHP supporti l'upload di files (Solitamente attivo di default)
\item L'estensione "mysqli" di PHP deve essere attiva (alcune distribuzioni di Linux la disattivano di default)
\item PHP deve avere supporto XML (testato con LibXML2), per il supporto a DOMDocument (Solitamente attivo di default)
\end{itemize}

\subsection{Installazione Base}

\subsection{Le pagine di errore (Opzionale - Solo Apache)}
Allo scopo di migliorare l'accessibilità del sito si mette a disposizione due pagine di errore personalizzate, ma per attivarle è necessario eseguire un semplice passaggio.\\
Nell'archivio del progetto, sottocartella "files" si mette a disposizione un file "htaccess" che dovrà essere messo nella root del server e rinominato ".htaccess" (attenzione al punto iniziale) allo scopo di configurare Apache per usare la pagina 404 che abbiamo creato.
